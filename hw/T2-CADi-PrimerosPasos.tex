\documentclass[letterpaper,8pt]{article}
\usepackage[activeacute,french,english,spanish]{babel}
\usepackage[utf8]{inputenc}
\usepackage{textcomp}
\usepackage[tmargin=1in,bmargin=1in,lmargin=1in,rmargin=1in]{geometry}
\usepackage{amsmath}
\usepackage{amsfonts}
\usepackage{multicol}
\usepackage{tocloft}
\usepackage{hyperref}
\usepackage{enumitem}
\usepackage{graphicx}% Include figure files

%%
\usepackage{float}
\usepackage{wrapfig}
%%%%%%
%\renewcommand{\theenumi}{\Alph{enumi}}
%%%%%%%%%%%%%%%%%%%%
%\setcounter{section}{-1}
\spanishdecimal{.}
\parindent=0pt
%%%%%%%%%%%%%%%%%%%%
\title{\vspace{-2.5cm}{\bf Programación en Julia: Primeros pasos}\\
%Ago-Dic 2019\\
\vspace{1cm}
Tarea 2. Funciones y control de flujo}
\date{}
%%%%%%%%%%%%%%%%%%%
\begin{document}

\maketitle
\thispagestyle{empty}
%%%%%%%%%%%%%%%%%%%
\section*{Aproximación numérica de una integral}
En este ejercicio aproximaremos el área bajo la curva de una función
(integrable) en una dimensión. La idea es tener una función que tome
como argumentos: la función a integrar y el intervalo de
integración. Nota: puede tener otros argumentos para tener precisiones
distintas del área.

Se evaluará:
\begin{enumerate}
\item La aproximación es con de sumas de Riemann.
\item La función que aproxima el área deberá aceptar cualquier otra,
  en una dimensión, que sea integrable para un intervalo finito.
\item La función debe especificar los tipos de las variables argumento.
\item Se deben incluir resultados de la evaluación de tres funciones
  distintas.
\end{enumerate}

%%%%%%%%%%%%%%%%%%%
\end{document}